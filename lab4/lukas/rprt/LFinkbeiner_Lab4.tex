\documentclass[12pt]{article}

\usepackage[margin=1in]{geometry}

\usepackage{amsmath, graphicx, caption, subcaption, xcolor}

\graphicspath{{res/}}

% I need a punchy title which is general enough
% to fit all three (if I can pull them off) points from the lab manual.
\title{\textcolor{red}{The Galactic Plane}}

\author{Lukas Finkbeiner}

\begin{document}

\maketitle

\begin{abstract}

% remember to motivate the paper

We study

1. velocity patterns of various H clouds using Doppler shifts

2. black hole mass

3. accuracy of a spiral arm fit

results, uncertainties.

\end{abstract}

\section{Introduction and Background}
% currently we need more connecting text

Given certain patterns, how well does a spiral fit? We need to control for the total number of data. We want the normalization of any two compared sets to be the same (so, we scale accordingly). What numpy functions I used.

21 cm line transition: 1420.405751786 MHz at rest.

\begin{equation} \label{eq:spiral}
R_\text{arm} = R_0 e^{\kappa(\phi - \phi_0)}
\end{equation}

The following equation is a conclusion of the tangent-point method for converting galactic rotation to Doppler velocity. $R_\odot \approx 220 $ km / s and $R_\odot \approx 8.5$ kpc $\equiv 2.6 \times 10^{17}$ km. 

\begin{equation} \label{eq:vel_dopp}
V_\text{Dopp} = \left[ \frac{V(R)}{R} - \frac{V(R_\odot)}{R_\odot} \right] R_\odot \sin(\ell)
\end{equation}

%\begin{equation}
%\frac{V_\text{Dopp}}{R_\odot \sin \ell} = \frac{V(R)}{R} - \frac{V(R_\odot)}{R_\odot}
%\end{equation}

%\begin{equation}
%\frac{V_\text{Dopp}}{R_\odot \sin \ell} + \frac{V(R_\odot)}{R_\odot} = \frac{V(R)}{R} = \frac{1}{R_\odot} \left[ \frac{V_\text{Dopp}}{\sin \ell} - V(R_\odot) \right] 
%\end{equation}

The following representation will be used in establishing the velocity curve inside the solar circle:

\begin{equation} \label{eq:vel_curve}
V(R) = \frac{R}{R_\odot} \left[ \frac{V_\text{Dopp}}{\sin \ell} - V(R_\odot) \right] 
\end{equation}

%This next form will be used in placing points outside the solar circle, to evaluate the spiral shape:

%\begin{equation}
%\frac{V_\text{Dopp}}{R_\odot \sin \ell} = \frac{V(R)}{R} - \frac{V(R_\odot)}{R_\odot}
%\end{equation}

%\begin{equation}
%\frac{V(R)}{R} = \frac{V_\text{Dopp}}{R_\odot \sin \ell} + \frac{V(R_\odot)}{R_\odot} 
%\end{equation}

%\begin{equation} \label{eq:outer_spiral}
%R = V(R) \left[ \frac{V_\text{Dopp}}{R_\odot \sin \ell} + \frac{V(R_\odot)}{R_\odot} \right]^{-1}
%\end{equation}

\textcolor{red}{You have to cite sources and find a better common unit.}

% ugh where does intro end and methods begin

Before we analyze the output of our spectrometer, we need to calibrate the spectra. The concepts are the same as \textcolor{red}{in lab 2 (this should be a citation?)}.

We want to change our local oscillator frequency between two values. This allows us to, via the mixer, adjust where the HI line appears in our spectra. We thereby create `on' (LO at 1270 MHz) and `off' (LO at 1268 MHz) spectra.

\begin{equation} \label{eq:line_shape}
T_\text{line} = G \, \frac{s_\text{on}}{s_\text{off}}
\end{equation}

The gain, G, is calculated with an additional spectrum ($cal$) that we deliberately contaminate with thermal noise. We do this by activating a noise diode which, for the vertical-polarization receiver (with which we are exclusively concerned here) is independently estimated to contribute about 80 Kelvins. Finally, we compare the magnitudes with a spectrum without this contamination and we compute:

\begin{equation} \label{eq:line_gain}
G = \frac{T_\text{sys, cal} - T_\text{sys, cold}}{\sum{(s_\text{cal} - s_\text{cold})}} \sum{s_\text{cold}}
\end{equation}

\begin{figure}
	\centering
	\includegraphics[width=.9\linewidth]{sig_chain}
	\caption{This is a simplification of the Leuschner signal chain, where the horizontal polarization has been ignored (we do not use those data in this report). Furthermore, we abstract-away the clock rate of the ROACH and the various stages of digitization. For our `on' LO frequency of 1270 MHz, the output spectra cover a frequency range 1415 to 1425 MHz.}
	\label{fig:sig_chain}
\end{figure}

As a demonstration of the Leuschner dish signal chain (illustrated in figure \ref{fig:sig_chain}), we show how we map a set of input signals to a final frequency axis for our plots (frequency versus brightness temperature). Since we are considering only frequency, this allows us to skip over the amplifiers and attenuators.

The initial band-pass filter ranges from 1.4 to 1.63 GHz. There is a down-converter that mixes this filtered signal with the signal from the local oscillator LO. This leads to a new frequency band (intermediate frequency IF) of 130 MHz to 360 MHz. Since we are using a real-valued mixer, we apply a band-pass filter to eliminate the sum frequencies. This filter is centered at 150 MHz and has a bandwidth of 10 MHz. Consequently, we expect the following frequency range in our outputs (which take the form of spectra in .fits files): a 10 MHz wide range beginning at 145 MHz IF (which maps to 1.415 GHz RF) and ending at 155 MHz IF (which maps to 1.425 GHz RF).

We used 1268 MHz for our second LO frequency, the `off' spectrum. This leads to an output range of 1.417 to 1.419 GHz.

We sample at 24 MHz (we keep 1/32 of the samples from the 768 MHz clock),->12MHz Nyquist bandwidth. so we are operating in the twelfth Nyquist window.

\section{Methods}

\quad \quad To observe the galactic plane, we have our trusty tracking script. After three labs' worth of development, it is almost correct!

How we calibrate: for each value $\ell$, we take four sets of ten spectra: ...

We convert from galactic to topocentric coordinates using rotation matrices, as described in a previous report. How much should I include? If I describe the matrices again, that will certainly have to go in the introduction.

We probably want a basic run-down of a .fits file, but perhaps some of that can go in the introduction.

\begin{figure}
	\centering
	\includegraphics[width=.9\linewidth]{1940_10_05_2020}
	\caption{}
	\label{fig:vis_demo}
\end{figure}

\section{Observations}

\quad \quad What would be a good $introduction$ to the data? We have hundreds of raw spectra. I am not sure how to $succinctly$ describe observations before we get into the analysis.

% This plot is simply a way of concisely representing the 260 spectra we collected over the interval $\ell \in [-9, 250]$ degrees (one-degree spacings).
\begin{figure}
	\centering
	\includegraphics[width=.9\linewidth]{Doppler_collection}
	\caption{We take the peak intensity of each spectrum to represent the shifted HI line for that galactic longitude.  Observe that, for our tangent-point considerations ($\ell \in [0, 90]$), we have an initial region featuring wild fluctuations followed by a relatively flat region for the second half of the angles. While this could represent physical realities, this seeming discontinuity of the derivative could also be a consequence of the primitive intensity-maximizing function, or it could be a consequence of pointing the telescope at angles close to mechanical boundaries.}
	\label{fig:Dopp_collection}
\end{figure}

\begin{figure}
	\centering
	\includegraphics[width=.9\linewidth]{tangentPoint_dispersion}
	\caption{This is an overplot of all 90 possible velocity curves. Specifically, when we treat the solar circle of the galaxy, we know that for each $\ell$, $R$ can take on a maximum value of $R_\odot$ (the radius of the solar circle) and a minimum value of $R_\odot \sin \ell$. The tangent-point method mandates the usage of $R_\text{min}$, but we here demonstrate how that method necessitates the collapse of line segments into points.}
	\label{fig:TP_disp}
\end{figure}

\begin{figure}
	\centering
	\includegraphics[width=.9\linewidth]{tangentPoint_selected_dispersions}
	\caption{Here we are essentially representing similar information to that of figure \ref{fig:TP_disp}. We use a semilog vertical axis to verify that there is one consistently-varying multiplicative offset ($\frac{1}{\sin \ell}$) as we examine different galactic longitudes. This plot also agrees with our expectation that our two boundary cases are meaningless: $\ell = 0$ degrees approaches an infinite linear velocity and $\ell = 90$ degrees corresponds to a velocity curve for which the minimum and maximum radii are the same (an infinitesimal segment).}
	\label{fig:TP_sel_disp}
\end{figure}

\section{Analysis}

\quad \quad I should explain the process of taking a spectrum to a point on our galactic plane map (Doppler shift calculations, then orthographic projection). 

We can estimate the mass of the super-massive black hole at the center of the milky way if we use Doppler velocities for our three closest angles ($\ell \in \{-1, 0, 1\}$ degrees) and integrate over radii from 0 to 44 million kilometers.

\section{Conclusions}

\quad \quad We seem to have at least a couple of results! How do I estimate the uncertainty on the Leuschner dish?

\section{Acknowledgments}

\quad \quad Who did what?

Theory and background provided by Aaron Parsons. ``LAB 4: Mapping the Galactic HI Line.'' Updated April 2020.

% Can we please get a bibliography this time?

\end{document}
