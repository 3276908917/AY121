\documentclass[a4paper]{article}

%% Language and font encodings
\usepackage[english]{babel}
\usepackage[utf8x]{inputenc}
\usepackage[T1]{fontenc}

%% Sets page size and margins
\usepackage[a4paper,top=3cm,bottom=2cm,left=3cm,right=3cm,marginparwidth=1.75cm]{geometry}

%% Useful packages
\usepackage{amsmath}
\usepackage{graphicx}
\usepackage[colorinlistoftodos]{todonotes}
\usepackage[colorlinks=true, allcolors=blue]{hyperref}

\usepackage{caption}
\usepackage{subcaption}

\graphicspath{{res/}}

\title{Lab 2}
\author{Lukas Finkbeiner}

\begin{document}
\maketitle

\begin{abstract}

This is a great place to have a to-do section, because I prefer to write abstracts at the end of the writing process.

* Every Python function that I write must have docstrings for full grade

* I need to pull from my repo to my lab computer by the time I submit my paper.

\quad * Ask Professor about the permissions thing (page 3), to make sure there are no technical difficulties.

\quad * Also ask the Professor what he recommends as to inclusion of docstrings in the lab report in the most clean, organized, and professional manner.

\end{abstract}

% No need to re-derive anything!!

\section{Introduction and Background}

Block diagram of the telescope electronics. Fortunately, it seems like the professor took care of most of this for you in lecture, February 11.

\section{Methods}

To place our test signal in the upper and lower sidebands, we use these two frequencies. To put the hydgrogen in the upper and lower sidebands, we set the first local oscillator to these other two frequencies.

\section{Results}

As a preface, I may want to include commentary on how I had to manually alter the data, based on readings from the oscilloscope, in order to account for the 

\section{Conclusions}

\section{Acknowledgments}

\textcolor{red}{You have to remedy your complete ignorance of BibTex}

\end{document}
