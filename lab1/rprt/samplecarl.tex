\documentstyle[11pt,aaspp4]{article}
\begin{document}

\title{Fun with TEX and LATEX}

\author{Carl Heiles\altaffilmark{1}}
\affil{Astronomy Department, University of California,
    Berkeley, CA 94720}

\altaffiltext{1}{email: cheiles@astro.berkeley.edu}

\begin{abstract}

	We illustrate some of the basic TEX features in a short paper.
If you want to know more, see Smith (1994).
\end{abstract}

\section{Generating a LATEX document} \label{nr3}

\subsection{Introduction (e.g. of a subsection) \label{subseven}}

	For this document to make sense, you should be looking not only
at the final, nicely-printed output, but also at the original file
(which is named ``ay120.tex'').  The use of the ``\verb"\"label''
command above means that you can refer to this section by its name
instead of its number.  This can be handy if you are composing a
document because you may wish to change the order of sections as you
proceed, but LATEX will always order things consecutively.  Thus, if you
type Section~\ref{nr3} the output will say ``Section 1'' because this
section, named ``nr3'', is the first section of this document.  If you
type Section~\ref{subseven}, then you are referring to this subsection.
If you type Table~\ref{equiltemp}, you are referring to the table at the
end.

\subsection{Embedded Equations}
	To do equations that are embedded within the written text,
enclose the equations in single dollar signs.  Thus, to write an
equation that says ``x is the integral of y over z dt'', you can write
$x = \int {y \over z} dz$.  The ``\verb"\"int'' means integral sign;
special characters, such as $\int$, are always preceeded by the
backslash character.  To make the backslash character actually appear in
the output, you have to use the type the special command ``\verb"\"''.
To make a fraction ``y over z'' you use the operator ``\verb"\"over'',
and you have to put curly brackets around that part of the equation you
want in the fraction.  You can also make {\bf boldface type}.  And, of
course, footnotes\footnote{Footnotes can be inserted like this.} are no
problem and are automatically numbered.

\paragraph{Special characters: e.g. of an emphasized paragraph}
	As you can see from the above paragraph, TEX has some special
characters such as curly brackets, dollar signs, and backslash, that it
uses for formatting commands. Of course, you may wish to have these
special characters actually appear in the output---for example, if you
are talking about the price of eggs, you want to use a dollar sign.
Generally, you can get the special characters to appear in the output by
preceding them with a backslash. We have the following special
characters:

	{\bf Curly brackets}, which are used to enclose text upon which
a certain operator should apply (such as ``\verb"\"over''); to make curly
brackets appear in the text, you must type \{ or \}.

	{\bf Dollar Sign}, used to denote math embedded in text. If you
want to specify the price of eggs, type \$1.73 per dozen.

	{\bf percent sign}, used to suppress processing of the rest of
the line. This makes it easy to insert comments to you, the editor of
the manuscript, that you want to see while you are editing but don't
want to appear in the final document. So if you want to say that your
grade on a lab report should be at least 90\% because of the effort
involved, you use the backslash.
% But the question is: should grades be assigned on effort alone?

	{\bf Ampersand, pound sign, ``at'' sign, double quotes, tilde,
carat, underscore}: \&, \#, \@, \", \~, \^, \_.

	You can construct just about every mathematical symbol with
appropriate commands. A typical Greek letter is written ``\verb"\"alpha'' and
looks like $\alpha$; it must be in math mode. You can offset equations
and number them as follows:

\begin{equation}
\label{AA1}
N(HI)_{TSAS, 19} = \int_a^b W_{s^\alpha} s^2 s^{x+y} ds
\end{equation}

\noindent Note that the ``\verb"\"label'' command words for equations, too.
%the ``\noindent'' keeps the line after the equation from being indended.
There is a huge list of mathematical symbols, which is appended to the
paper version of this document.

\section{Turning your file into the final output} \label{nr2}

	When you have generated your file you then must process it with
the LATEX command.  Undoubtedly you will encounter errors.  The output
will give you an all-too-brief description of the error, and most
importantly it will tell you the {\it line number} on the file where the
error occurs.  Mostly, your errors will probably consist of not having
matched dollar signs, having unmatched curly brackets, or using a
backslash with an operator that doesn't exist.  In the case of unmatched
symbols, the error may exist on some previous line; TEX may only realize
somewhat belatedly that there is an unmatched symbol.  The quickest way
to find your error is jump to the line number and work backwards.

	Running LATEX generates what's called a ``dvi'' file. If your
original file is called lab1.job, LATEX will produce a file called
lab1.dvi. To view this file on the terminal screen,
as it will appear when printed, type ``xdvi
lab1.dvi''. To print the file, type ``dvips lab1.dvi''.

\clearpage
%the ``clearpage'' command starts a new page.

\begin{references}

\reference{} Smith, F. 1994, {\it Safe Tex}, Harcourt Brace Co.

\end{references}

;\clearpage

\figcaption{Equilibrium temperature of TSAS gas for various conditions
as described in section~\ref{nr2}.  \label{equiltemp}}
%Note that the ``\label'' command works for figure captions, too.


\begin{deluxetable}{crrcrrl}
%the ``crrcrrl'': there are 7 columns. ``c'' means the information
%in a column is centered; ``r'', right-justified; ``l'', left-justified.
\footnotesize
\tablecaption{This is the TITLE; note the label command! \label{2abs}}
\tablewidth{430pt}
%To center the numerical part of the table with respect to the comments
%that come afterwards, you will have to experiment with changing the
%tablewidth. For example, if you use \tablewidth{430pt}, this table
%will look centered. The LATEX output will tell you some reasonable numbers.

%The following gives the headings of the seven tables. Note that each
%column is separated by an ampersand. Note that the column labels are
%enclosed in curly brackets.
\tablehead{
\colhead{Source} & \colhead{$\ell$} & \colhead{$b$} &
\colhead{$\tau_{max}$} &
\colhead{$v_{LSR}$} & \colhead{FWHM} & \colhead{ref, note}
}
%Now we start the data. Each column's contents is separated by an ampersand.
%Each row is terminated with the ``\nl'' (newline) command.
%It is NOT NECESSARY to align the columns in this text file!
%ONE THING, THOUGH: DON'T LEAVE ANY BLANK LINES IN THE DATA SECTION
%(i.e., between \startdata and \enddata).

\startdata
0624-058 (3C161) & 215.4 & --8.0  &    0.67      &  12.0 &   4.5 &   1,a
\nl
3C161            & 215.4 & --8.0  &   0.88       &   7.6 &   2.5 &   1,a
\nl
3C161(OH)        & 215.4 & --8.0  &  0.013       &   8.6 &   1.2 &   3
\nl
PKS0605-08       & 215.7 & --13.5 &
0.80$^b$         &  7.3  &   8.9 &   2
\nl
0530+04 (4C04.18)& 200.0 & --15.3 & 0.8:         &  4.3: &   6.7:&   2
\nl
3C135            & 200.5 & --21.0 & $\lesssim 0.11$&\nodata &\nodata & 2
\nl
PKS0533-12       & 215.4 & --22.2 & 0.36         &  3.9  &   8.0 &    2
\nl
\enddata

\tablerefs{(1) Mebold {\it et al.} (1981), Mebold
{\it et al.} (1982); (2) Crovisier, Kaz\`es, and Aubrey (1978);
(3) Dickey, Crovisier, and Kaz\`es (1981).}

\tablenotetext{a}{Mebold {\it et al.} (1982) list 3 components in
addition to the 4 listed here.}

\tablenotetext{b}{We have not listed a second, weaker Gaussian component
because of poor signal/noise.}

\tablecomments{This comment applies to the whole table and you can
put it either in front or behind the other comments. This is a good place
to explain your columns: Column 1 is the source name; columns 2 and 3 are
the Galactic coordinates in degrees; Column 4 is the opacity; Column 5
is the Velocity with respect to the LSR in km s$^{-1}$; Column t is the
FWHM of the line in km s$^{-1}$. }

\end{deluxetable}


\end{document}
